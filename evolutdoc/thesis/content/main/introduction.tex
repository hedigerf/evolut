%
% Introduction
%

% !TEX root = ../../main.tex

\chapter{Einleitung}

  Viele Optimierungsprobleme,
  die in der Informatik und Robotik auftreten sind mit mathematischen Methoden nicht innert nützlicher Frist lösbar.
  Oft spielt dabei die Anzahl der Freiheitsgrade eine entscheidende Rolle.
  Die Anzahl der Freiheitsgrade und somit der Menge möglicher Lösungen übersteigt die Kapazität aktueller Computer.
  \\
  Ein alternatives Verfahren findet sich in der Natur. In der Natur werden ständig Optimierungsprobleme gelöst.
  Das Prinzip dabei ist, dass die Individuen überleben und Nachwuchs zeugen,
  die am besten an die aktuelle Umgebung angepasst sind.
  Die nächste Generation ist besser an die Bedinungen ihrer Vorgänger angepasst. % ZITAT
  \\
  Mit dieser Methode können Optimierungsprobleme auf einem Computer gelöst werden,
  da nur ein Teil der ganzen Lösungsmenge berechnet wird.
  Auf Computersystemen kann dieser Prozess virtuell abgebildet und im Schnelllauf durchg eführt werden.
  \\
  % ea -> approx lsg

  \section{Ausgangslage}

    \todo[inline]{Nennt bestehende Arbeiten/Literatur zum Thema -> Literaturrecherche}
    \todo[inline]{Stand der Technik: Bisherige Lösungen des Problems und deren Grenzen}

    Evolutionäre Algorithmen sind bereits in vielen Bereichen im Einsatz.
    \\
    Die NASA verwendet [..]~\cite{Hornby2006}.
    \\
    Theoretische Grundlagen

    Das notwendig theoretisch Wissen zur Bewältigung dieser Arbeit wurde mit Hilfe von den Buechern
    Bio-inspired artificial intelligence: theories, methods, and technologies~\cite{book:bioInspired} und
    Evolutinäre Algorithmen~\cite{book:evAlgo} erarbeitet.

    % Altes Problem der Robotik.

\section{Zielsetzung}
  Das Ziel dieser Arbeit ist sechsbeinigen künstlichen Tieren, mit Hilfe eines evolutionären Alogirthmus, in einer zweidimensionalen Umgebung das Gehen zu lernen.
  Um dieses Ziel zu erreichen, soll eine Applikation erstellt werden, welche den evolutinären Algorithmus implementiert.
  Die Applikation muss ebenfalls die Möglichbeit bieten, eine Simulation durchzuführen.
  Es soll nicht nur die Steuerung der Beine evolviert werden (Gehen lernen), sodern auch die Form des Körpers und die Abmassung der Beine.

  \begin{comment}
    Als Mass für die Fitness wird eine Funktion der Zeit verwendet, welche das Tier braucht,
    um einen gegebenen Parcours zurückzulegen.
    Dabei müssen Individuen berücksichtigt werden, die den Parcours nicht vollständig ablaufen können.
    Die Fitnessfunktion berücksichtigt in diesem Fall die zurückgelegte Strecke.
    \\
    Als Randbedingungen vorgegeben sind:
    \begin{enumerate}
      \item Forderung einer physikalisch sinnvollen Bewegung
      \item Obere Grenze für die total aufgewandte Energie
      \item Beschränkung der abgegebenen Leistung
    \end{enumerate}
    Damit die Forderung nach einer physikalischen Bewegung erfüllt werden kann,
    wird die Bewegung mit Hilfe einer Physik-Engine simuliert.
    Diese Engine wird von den Studierenden ausgewählt und kann als Blackbox verwendet werden.
  \end{comment}
  \subsection{Requirements}

    \subsubsection{Fitnessfunktion}
      In frühen Generation kann es vorkommen, dass die künstlichen Tiere sich keinen einzigen Schritt weit während der Simulation
      fortbewegen können. Die Zeit, welche ein Individuum für die Bewältigung eines Parcours (\vref{subsub:parcours}) brauchen würde, wäre somit unendlich.
      Um dieses Problem zu umgehen, wird als Mass für die Fitness die zurückgelegte Strecke verwendet.
      Für die Bewältigung des Parcours steht dem Individuum eine fixe Anzahl Zeiteinheiten zur Verfügung.

    \subsubsection{Parcours\label{subsub:parcours}}
      Die Individuen müssen während der Simulation eine Strecke zurücklegen.
      Diese Strecke und ihre Begebnheiten werden als Parcours bezeichnet.
      \vref{fig:scetchParcours} zeigt eine Skizze des Parcours.
      Mit zunehmenden Generationen soll die Schwierigkeit des Parcours steigen.
      Als Schwierigkeit kann man die maximale Steigung des Parcours bezeichnen.
      Der Parcours soll zufällig generiert werden.

      \begin{figure}[H]
        \includegraphics[scale=1]{graphics/scetch_parcours}
        \caption{Skizze Parcour\label{fig:scetchParcours}}
      \end{figure}


    \subsubsection{Limiten}
      Limiten verhindern das der Parcours durch triviale Lösungen wie z.B. unendlich grosse Beine bewältigt werden kann.
      Für folgende Komponenten werden Limiten eingeführt:
      \begin{itemize}
        \item Winkelgeschwindigkeit auf den Gelenken
        \item Beinhöhe
        \item Beinbreite
        \item Masse des gesammten Individuums
        \item Fläche des Körpers
      \end{itemize}

    \subsection{Projekt Management}
      Die verschiedenen Arbeitspackete werden eine Woche im vorraus geplant.
      Dieses Vorgehen wird auch als rollende Planung bezeichnet.
      Ein Projektvorgehensmodell wird nicht benutzt, jedoch ist eine Projektplanung mit Hilfe von MS Project 2016
      erstellt worden. Alle wichtigen Meilensteine finden sich im Projektplan wieder.
      Der Stand der Arbeit wurde jeweils an einem wöchentlichen Meeting mit den Betreuern besprochen.
      Das Protokoll wurde vor dem nächsten Meeting jeweils verschickt.
