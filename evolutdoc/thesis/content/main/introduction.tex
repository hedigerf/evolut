%
% Introduction
%

% !TEX root = ../../main.tex

\chapter{Einleitung}

Evolutionäre Algorithmen helfen Problemstellung zu lösen, welche sonst nicht ohne so mathematisch lösbar sind.
Das heisst, dass der Problemraum zu gross für gängige mathematische Lösungsvorgehen ist.
Diese Arbeit hat sich das Ziel gesetzt, sechs Beinige Individuen und ihre Steuerung zu evolvieren.

\section{Ausgangslage}

Das notwendig theoretisch Wissen zur Bewältigung dieser Arbeit wurde mit Hilfe von den Buechern
Bio-inspired artificial intelligence: theories, methods, and technologies \cite[]{book:bioInspired} und
Evolutinäre Algorithmen \cite[]{book:evAlgo} erarbeitet.
  \todo[inline]{Nennt bestehende Arbeiten/Literatur zum Thema -> Literaturrecherche}
  \todo[inline]{Stand der Technik: Bisherige Lösungen des Problems und deren Grenzen}

\section{Zielsetzung}
  Das Ziel dieser Arbeit ist sechsbeinigen künstlichen Tieren, mit Hilfe eines evolutionären Alogirthmus, das Gehen zu lernen.
  Um dieses Ziel zu erreichen, soll eine Applikation erstellt werden, welche den evolutinären Algorithmus implementiert.
  Die Applikation muss ebenfalls die Möglichbeit bieten, eine Simulation durchzuführen.
  Es soll nicht nur die Steuerung der Beine evolviert werden (Gehen lernen), sodern auch die Form des Körpers und die Abmassung der Beine.

  \begin{comment}
    Als Mass für die Fitness wird eine Funktion der Zeit verwendet, welche das Tier braucht,
    um einen gegebenen Parcours zurückzulegen.
    Dabei müssen Individuen berücksichtigt werden, die den Parcours nicht vollständig ablaufen können.
    Die Fitnessfunktion berücksichtigt in diesem Fall die zurückgelegte Strecke.
    \\
    Als Randbedingungen vorgegeben sind:
    \begin{enumerate}
      \item Forderung einer physikalisch sinnvollen Bewegung
      \item Obere Grenze für die total aufgewandte Energie
      \item Beschränkung der abgegebenen Leistung
    \end{enumerate}
    Damit die Forderung nach einer physikalischen Bewegung erfüllt werden kann,
    wird die Bewegung mit Hilfe einer Physik-Engine simuliert.
    Diese Engine wird von den Studierenden ausgewählt und kann als Blackbox verwendet werden.
  \end{comment}
  \subsection{Requirements}

    \subsubsection{Fitnessfunktion}
      In frühen Generation kann es vorkommen, dass die künstlichen Tiere sich keinen einzigen Schritt weit
      fortbewegen können. Die Zeit, welche ein Individuum für die Bewältigung eines Parcours (\vref{subsub:parcours}) brauchen würde, wäre somit unendlich.
      Um dieses Problem zu umgehen, wird als Mass für die Fitness die zurückgelegte Strecke verwendet.
      Für die Bewältigung des Parcours steht dem Individuum eine fixe Anzahl Zeiteinheiten zur Verfügung.
      
    \subsubsection{Parcours\label{subsub:parcours}}
      Es muss mehr als nur einen Parcours für die Evolution der Individuen geben,
      ansonsten werden die Individuen nur auf den einen Parcours evolviert.
      Als Start-Parcours sollte ein möglichst einfach zu bewältigender Parcours dienen.
      Die Schwierigkeit des Parcours soll mit zunehmenden Generationen steigen und zufällig generiert werden.



    \subsubsection{Obergrenze für die totale aufgewandte Energie}

      Damit das Problem nicht trivial gelöst werden kann
      und ein Individuum einfach ins Unendliche beschleunigt werden kann,
      wird der Anspruch nach einer Obergrenze für die total aufgewandte Energie gestellt.
      Somit muss die abgebende Leistung beschränkt werden. Es werden Limiten eingeführt,
      um dieses Problem zu lösen.

    \subsection{Projekt Management}
      Der Stand der Arbeit wird jeweils an einem wöchentlichen Meeting mit den Betreuern besprochen.
      Das Protokoll wird vor dem nächsten Meeting jeweils verschickt.
      Ein Projektvorgehensmodell wird nicht benutzt, jedoch ist eine Projektplanung mit Hilfe von MS Project 2016
      erstellt worden. Die verschiedenen Arbeitspackete werden eine Woche im vorraus geplant.
      Dieses Vorgehen wird auch als rollende Planung bezeichnet. Alle wichtigen Meilensteine finden sich im Projektplan wieder.
