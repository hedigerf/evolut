%
% Perspective
%

% !TEX root = ../../main.tex

\chapter{Diskussion und Ausblick}

  \todo[inline]{Bespricht die erzielten Ergebnisse bezüglich ihrer Erwartbarkeit, Aussagekraft und Relevanz}
  \todo[inline]{Interpretation und Validierung der Resultate}
  \todo[inline]{Rückblick auf Aufgabenstellung, erreicht bzw\@. nicht erreicht}
  \todo[inline]{Legt dar, wie an die Resultate (konkret vom Industriepartner oder weiteren
  Forschungsarbeiten; allgemein) angeschlossen werden kann; legt dar, welche Chancen die
  Resultate bieten}

  \section{Ausblick}
    Die fehlende Implementation der Reaktion auf das Feedback der Steuerung ist sicher die Komponente, welche am meisten helfen würde, bessere Resultate zu finden.
    Ein weiterer Punkt ist die langsame und teilweise fehlerhafte Physik-Engine p2.js. Das Ersetzen der Physik-Engine wäre der zweitwichtigste Schritt um bessere Resultate zu finden.
    Mit einer besseren Physik-Engine könnte auch schneller simuliert werden.
