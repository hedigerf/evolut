%
% Abstract
%

% !TEX root = ../main.tex

\chapter{Abstract}

  Optimization of many practically relevant and
  interesting problems in informatics and robotics are expensive tasks with respect to computing resources.
  Solving these problems requires expertise and costs time and resources.
  If a solution space grows too complex, formal solution methods reach a limit.
  Evolutionary algorithms produce through trial and error new approximative solutions.
  Following the principle of the survival of the fittest, these approximations should improve over time.

  \smallskip

  In this paper we try to teach artificial animals with six legs how to move through a parcour.
  It is explored how the geometry and movement sequence can be evolved.
  Afterwards, an analysis is done on how the movement sequence and the shape of an evolved individual looks.
  We postulate a number of hypotheses 1,2,3. % TODO auflisten der hypothesen, f-fragen

  main research questions where: \ldots

  The hypotheses are examined based on the results of different simulations.

  \smallskip

  As a main result we show that three different types of movement sequences were developped.
  A jump, row and roll movement. Animals which use a rolling as movement to have a spherical body.
  When animals use rowing, their body resembles a flat shape. However, animals which jump don't show a clear trend.
  As further optimization, a feedback system can be utilized to tackle ascending and descending parts of a parcour.
